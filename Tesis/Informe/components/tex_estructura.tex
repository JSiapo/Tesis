\section{Estructura de la tesis}

{\bf Ejemplo:}\par
\vskip 0.1cm
El presente trabajo está dividido en cinco capítulos. El primer capítulo presenta los aspectos generales de la investigación realizada tal como justificación, formulación del problema, hipótesis, los objetivos y la estructura de la tesis.

En el capítulo dos se presenta el referencial teórico, soporte del tema, contemplando los conceptos de sustentabilidad urbana, logística directa y reversa, modelamiento y ruteo. Finalmente el método empleado en la investigación.

El tercer capítulo trata del tema central de la tesis, diseñandose los modelos respectivos propuestos...   

 En el cuarto capítulo se presentan los resultados y discusión obtenida en la investigación. En el capítulo cinco se presentan las consideraciones finales obtenidas en esta tesis. Inicialmente se presentan las conclusiones, seguida de las recomendaciones para futuras investigaciones relacionadas al tema en cuestión.

Finalmente las referencias bibliográficas usadas para la investigación en esta tesis y los anexos donde se presentan los programas elaborados y en apéndice un pequeño glosario de ciertos términos usados en esta investigación. Finalmente la declaración jurada y autorización de la tesis.

