\section{Objetivos}
Es necesario establecer qué pretende la investigación, es decir, cuáles son sus objetivos. Hay investigaciones que buscan contribuir a resolver un problema en especial, y otras tienen como objetivo principal probar una teoría o aportar evidencia empírica en favor de ella. \par 
\vskip 0.3cm
Segun, los objetivos tienen que expresarse con claridad para evitar posibles desviaciones en el proceso de investigación y deben ser susceptibles de alcanzarse; son las guías del estudio y hay que tenerlos presentes durante todo su desarrollo. Los objetivos deben ser congruentes entre sí.
\vskip 0.3cm
Describir el objetivo central o propósito del proyecto de investigación (debe estar alineado con el problema e hipótesis), así como los objetivos específicos, los cuales deben reflejar los cambios que se esperan lograr en trabajo de tesis (variables). Para estos objetivos específicos utilice verbos como: describir, indicar, modificar, controlar, producir (tecnologías), recuperar, etc..

\subsection{Generales}
Debe explicitar lo que se espera lograr con el estudio en términos de conocimiento. Debe dar una noción clara de lo que se pretende describir, determinar, identificar, comparar y verificar.


\subsection{Específicos}
Son la descomposición y secuencia lógica del objetivo general. Son un anticipo del diseño de la investigación.
\vskip 0.3cm



{\bf Ejemplo de objetivos:}\\
{\bf Objetivo General:}
\begin{enumerate}
\item[a)] Item1.
\vskip 0.3cm
\item[b)] Item2.
\end{enumerate}
\vskip 0.2cm
{\bf Objetivos específicos:}
\begin{enumerate}
\item[a)] Item1.
\item[b)] Item2.
\end{enumerate}
