\section{Hipótesis}
Preferentemente para investigaciones explicativas debe ser una respuesta a priori y tentativa guardando coherencia con el problema científico, se formula como una proposición afirmativa, con un lenguaje claro y específico.  Las hipótesis se obtienen por deducción lógica y está sustentada en los conocimientos científicos. \par  
\vskip 0.3cm
{\bf Criterios para formular hipótesis:} \cite{Erica}
\begin{enumerate}
\item[a)] Toda hipótesis de investigación debe ser verificable estadísticamente.  Puede ser difícil o imposible de verificar porque no existe un conocimiento sobre el cual se pueda formular una hipótesis, o bien, porque una o más variables no son medibles.
\vskip 0.2cm
\item[b)] Toda hipótesis debe indicar la relación entre variables, lo que implica que las variables deben ser medibles.
\vskip 0.2cm
\item[c)] Toda hipótesis debe tener sus límites. Pueden escogerse hipótesis que sean sencillas de validar, y sin embargo, altamente significativas.
\vskip 0.2cm
\item[d)] El investigador debe tener una razón específica para considerar una hipótesis, ya sea teórica o por alguna evidencia concreta.    
\end{enumerate}
