%%%%%%%%%%%%%%%%%%%%%%%%%%%% RESUMEN%%%%%%%%%%%%%%%%%%%%%%
\newpage
\begin{center}
 \addcontentsline{toc}{chapter}{Resumen}
 {\bf\LARGE Resumen}
\end{center} 
\vskip 0.5cm
\begin{quotation}
{\bf Ejemplo:}\par

La investigación bibliográfica revela una preocupación de los gobiernos en lo relacionado al destino final de los residuos sólidos urbanos (RSU), con el objetivo de preservar la salud de la población, el medio ambiente urbano y rural. En este contexto y para el caso de las ciudades peruanas se esperaba que con la desactivación legal de los botaderos hasta el año 2020, surgiesen medidas que viabilicen la colecta selectiva, reciclaje y reutilización para aproximadamente el $80\%$ del volumen total de residuos colectados y destinados a locales no apropiados. 
\vskip 0.2cm 
En este sentido esta investigación tiene como objetivo principal modelar y planificar una red de logística reversa para una región urbana, dimensionando el flujo de RSU que será movido a lo largo de la red, el número y capacidad de las estaciones de colecta, de las unidades productivas y especiales necesarias para su colecta, transporte y disposición final. Los resultados muestran que es posible realizar un modelo matemático para este tipo de problemas, así como su aplicación en diversas regiones sin necesidad de grandes cambios en el modelo propuesto.    

\vskip 0.3cm
\hspace*{-0.6cm}{\bf Palabras claves:} residuos sólidos urbanos, logística reversa, modelo matemático.
\end{quotation}
%%%%%%%%%%%%%%%%%%%%%%%%%%%%%%%%%%%%%%%%%%%%%%%%%%%%%%%%%%%%%%%%%%%%%%%%%%%%%%%%%%%%


%%%%%%%%%%%%%%%%%%%%%%%%%%%%ABSTRACT%%%%%%%%%%%%%%%%%%%%%%
\newpage
\begin{center}
 \addcontentsline{toc}{chapter}{Abstract}
 {\bf\LARGE Abstract}\vskip 1.5cm
\end{center} 
\begin{quotation}

{\bf Ejemplo:}\par

The literature reveals a concern of Governments with the disposal of municipal solid waste (MSW) in order to preserve the health of the population, the urban and rural environment. In this context and for the case of peruvian cities, it was expected that, with the legal command for the deactivation of landfills by 2020, measures would be adopted in order to enable the selective collection, recycling and reuse for about $80\%$ of the total volume of collected solid waste and intended to inappropriate places. 
\vskip 0.2cm
In this sense, this research aims to model and plan a reverse logistics network to an urban area, dimensioning the flow of MSW that will be moved along the network, the number and capacity of collection stations, and the productive and special units required for their collection, transportation and final disposal. The results show to be possible perform mathematical modeling of this problem with low investment, as well as apply it in various regions without major changes in the proposed model.

\vskip 0.3cm
\hspace*{-0.6cm}{\bf Keywords:} solid waste, reverse logistics, mathematical modeling.
\end{quotation}
%%%%%%%%%%%%%%%%%%%%%%%%%%%%%%%%%%%%%%%%%%%%%%%%%%%%%%%%%%%%%%%%%%%%%%%%%%%%%%
